\documentclass{article}
\usepackage[utf8]{inputenc}
\usepackage{courier}
\usepackage{amsmath}
\usepackage{amssymb}
\usepackage{listings}
\usepackage{xcolor}
\usepackage{algorithm}
\usepackage[noend]{algpseudocode}
\usepackage{booktabs}
\usepackage{amsmath}
\usepackage[margin=1in]{geometry}
\usepackage{float}

\title{Penn CIS5300 - Natural Language Processing - Sentiment Classification}
\title{Thumbs up? Sentiment Classification using Machine Learning
Techniques - Bo Pang, Lillian Lee, Shivakumar Vaithyanathan}
\author{Notes: PENN CIS5300 Jonathon Delemos - Dr. Chris Callison Burch}
\date{\today}

\begin{document}

\maketitle

\subsection{Abstract}

Throughout this paper we will investigate the Naive Bayes, maximum entropy, and support vector machine learning methods. We will discover that
the factors that exist in sentiment classifcation make the problem quite challenging.

\subsection{Formulas - Naive Bayes}

This is the Naive Bayes formula for calculating the probability that document $d$ belongs to class $c$:

\[
    P_{\text{NB}}(c \mid d) := \frac{P(c) \cdot \left( \prod_{i=1}^{m} P(f_i \mid c)^{n_i(d)} \right)}{P(d)}
\]

Let’s define the components:

\begin{itemize}
    \item $P(c)$: Prior probability of class $c$
    \item $f_i$: The $i^{\text{th}}$ word (feature) in the vocabulary
    \item $n_i(d)$: Number of times feature $f_i$ appears in document $d$
    \item $P(f_i \mid c)$: Probability of word $f_i$ given class $c$
    \item $\prod_{i=1}^{m}$: Product over all $m$ words in the vocabulary
    \item $P(d)$: Probability of document $d$ (used to normalize; often ignored in $\arg\max$)
\end{itemize}

This is a formula we are quite familiar with. We are searching for the probability of class given document. This is the positive or negative
sentiment given a particular document. We take the probability of the word raised to the number of times it appears in the document. Then we take that product and
reproduce it against every word in the document, also multiplying against the probability of the class and divided by the probability of the document.

\subsection{Maximum Entropy}

Maximum entropy classification (MaxEnt, or ME,
for short) is an alternative technique which has
proven effective in a number of natural lan-
guage processing applications (Berger et al., 1996).
Nigam et al. (1999) show that it sometimes, but not
always, outperforms Naive Bayes at standard text
classification. It excels by making less assumptions about the data. Its estimate of P (c | d) takes the following exponential form:

\[
    P_{\text{ME}}(c \mid d) := \frac{1}{Z(d)}exp(\sum_{i} \lambda_{i,c} F_{i,c}(d,c))
\]

Let's define the terms.

\begin{itemize}
    \item $P_{\text{ME}}(c \mid d)$: Probability that document $d$ belongs to class $c$ under the Maximum Entropy model
    \item $Z(d)$: Partition function or normalization constant, ensures probabilities across all classes sum to 1
    \item $\exp$: Exponential function $e^x$
    \item $\sum_i$: Sum over all features indexed by $i$
    \item $\lambda_{i,c}$: Weight parameter associated with feature $i$ and class $c$, learned during training
    \item $F_{i,c}(d, c)$: Feature function indicating the presence, count, or strength of feature $i$ in document $d$ for class $c$
\end{itemize}

\subsection{Support Vector Machines}

Support vector machines have been shown to be highly effective.  They are large-margin, rather than proba-
bilistic, classifiers, in contrast to Naive Bayes and
MaxEnt. In the two-category case, the basic idea be-
hind the training procedure is to find a hyperplane,
represented by vector ~w, that not only separates
the document vectors in one class from those in the
other, but for which the separation, or margin, is as
large as possible. This search corresponds to a con-
strained optimization problem; letting $c_j \in {1,-1}$
(corresponding to positive and negative) be the correct class of document dj , the solution can be written
as

\[
    \vec{w} := \sum_j \alpha_j c_j \vec{d}_j, \quad \alpha_j \geq 0
\]

Let's define the terms.

\begin{itemize}
    \item{$\alpha_{j}$ - are learned coefficients}
    \item $c_j$ - class model for training model (j+1)
    \item $d_j$ - feature training model for document \textit{j}
\end{itemize}

\textbf{Big Idea}: This is how the weight vector $\vec{w}$ is built.


\subsection{Comparison of Formulas}


\textbf{Naive Bayes} performs the worst, while \textbf{Support Vector Machines} perform the best. It should be noted that bigrams did not greatly improve the efficiency
of the algorithms. Furthermore, bigrams rendered algorithms like NB useless, as conditional dependence was required.

\subsection{Summary}

Throughout this article we compared human classification of sentiment using word selection (60 percent accuracy)
with three simple sentiment classification algorithms. We found that the sentiment detection algorithms outperformed
the human word selections by rouhgly 20 percent. All trials were more than fifty percent accurate.

\subsection{Questions}

I've seen the math for both the Maximum Entropy and Support Vector Machines, but how does it work in practice? I'd
be intersted to see a problem worked out.


\end{document}