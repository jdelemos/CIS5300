\documentclass{article}
\usepackage[utf8]{inputenc}
\usepackage{courier}
\usepackage{amsmath}
\usepackage{amssymb}
\usepackage{listings}
\usepackage{xcolor}
\usepackage{algorithm}
\usepackage[noend]{algpseudocode}
\usepackage{booktabs}
\usepackage{amsmath}
\usepackage[margin=1in]{geometry}
\usepackage{float}

\title{CIS5300 - Speech and Language Processing - Homework 1 Notes}
\author{Jonathon Delemos - Mark Yakstar}
\date{\today}

\begin{document}

\maketitle

\subsection{Regular Expressions (RE) Tools}

The assignment highlights three core python regex functions: 
\begin{itemize}
    \item re.search() \textit{Finds the first substring with the given pattern}
    \item re.findall() \textit{Extracts all non-overlapping matches.}
    \item re.sub() \textit{Takes one string and replaces with the desired one.}
\end{itemize}

Regular expressions are a powerful tool that should be used to transform text, locate patterns, extract structured information.

\subsection{Data Structures}

This assignment relied heavily on python dictionaries and lists to manipulate data at various levels. 
\newline 
\textbf{Dictionaries}
\begin{itemize}
    \item word freqs = {} - keys are immutable strings, values are integers
    \item loop through requires items(), \textit{for word, count in word freq.items():} 
    \item return {word: count for word, count in freq.items() if count $>$= threshold} List comprehensions combined with dictionaries were prevalent 
\end{itemize} 
\subsection{Questions}

It would be nice to speak with the professor. His lectures aren't posted anywhere..

\end{document}
