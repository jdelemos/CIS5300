\documentclass[12pt]{article}
\usepackage[a4paper,margin=1in]{geometry}
\usepackage{booktabs}
\usepackage{longtable}
\usepackage{caption}
\usepackage{graphicx}
\usepackage{amsmath}
\usepackage{hyperref}
\usepackage{array}
\usepackage{setspace}
\usepackage{float}

\title{\textbf{CIS530: Natural Language Processing}\\
Homework 5: Advanced Vector Space Models}
\author{Jonathon Delemos\\
University of Pennsylvania}
\date{}

\begin{document}
\maketitle
\begin{abstract}
This report presents the results and analysis for CIS530 Homework 5, which focuses on semantic similarity and word clustering using both sparse and dense vector representations. Tasks involve comparing human judgment scores from the SimLex-999 dataset against word embeddings, computing Kendall’s $\tau$ correlations, and evaluating clustering performance on paraphrase sets. The experiments include random baselines, sparse co-occurrence vectors, and dense Word2Vec embeddings trained on Google News data.
\end{abstract}

\section*{SimLex-999 Dataset Revisited}

\subsection*{Least Similar Pairs (2.1)}

\textbf{Question:} What are the least similar two pairs of words based on human judgement scores and vector similarity? Do the pairs match?

\begin{table}[H]
\centering
\caption{Least Similar Word Pairs}
\begin{tabular}{@{}lll@{}}
\toprule
\textbf{Source} & \textbf{Word Pair} & \textbf{Score} \\
\midrule
Human Judgement & SHRINK, GROW & 0.23 \\
                & NEW, ANCIENT & 0.23 \\
Vector Similarity & HOUSE, KEY & -0.0413 \\
                 & FLOWER, ENDURANCE & -0.0368 \\
\bottomrule
\end{tabular}
\end{table}

\noindent
\textbf{Discussion:}  
The answers do not match. Human scores in SimLex-999 range roughly from 0–10, while cosine similarities range from about -0.1 to 1.0. The mismatch (action antonyms vs. noun-noun relations) is expected since the ranges differ due to the cosine application. Human judgments emphasize semantic polarity, while embeddings reflect contextual co-occurrence.

\subsection*{Most Similar Pairs (2.2)}

\textbf{Question:} What are the most similar two pairs of words based on human judgement scores and vector similarity? Do the pairs match?

\begin{table}[H]
\centering
\caption{Most Similar Word Pairs}
\begin{tabular}{@{}lll@{}}
\toprule
\textbf{Source} & \textbf{Word Pair} & \textbf{Score} \\
\midrule
Human Judgement & QUICK, RAPID & 9.7 \\
                & VANISH, DISAPPEAR & 9.8 \\
Vector Similarity & SOUTH, NORTH & 0.967 \\
                 & NORTH, WEST & 0.941 \\
\bottomrule
\end{tabular}
\end{table}

\noindent
\textbf{Discussion:}  
No, the pairs for mathematical vector scores and human judgement scores do not match. The most similar words according to human scores are action verbs, while the vector math suggests that directional adjectives are most similar. This discrepancy illustrates that embeddings capture contextual association rather than strict synonymy or semantic equivalence.

\subsection*{Correlation Scores and Model Comparison (2.3)}

\begin{table}[H]
\centering
\caption{Kendall's $\tau$ Correlations for GloVe Models}
\begin{tabular}{@{}lccc@{}}
\toprule
\textbf{Model} & \textbf{Dimensions} & \textbf{$\tau$} & \textbf{p-value} \\
\midrule
GloVe 6B 50D & 50 & 0.1810 & 1.22e-17 \\
GloVe 6B 100D & 100 & 0.2051 & 3.41e-22 \\
GloVe 6B 200D & 200 & 0.2367 & 4.99e-29 \\
GloVe 6B 300D & 300 & 0.2589 & 2.08e-34 \\
GloVe 840B 300D & 300 & 0.2861 & 1.29e-41 \\
\bottomrule
\end{tabular}
\end{table}

\noindent
\textbf{Discussion:}  
Increasing embedding dimension from 50D to 300D and corpus size from 6B to 840B improves Kendall’s $\tau$, indicating that richer embeddings capture human similarity judgments better. There is a clear positive correlation between larger corpora and higher $\tau$ values. All correlations are statistically significant, confirming that the results are not due to chance. However, improvements diminish as dimension increases, revealing diminishing returns beyond 300 dimensions.

\section*{Clustering Experiments}

\subsection*{Task 3.1: Cluster Randomly (1 point)}

\begin{longtable}{@{}lcc@{}}
\caption{Task 3.1: Random Clustering Results}\\
\toprule
\textbf{Target} & \textbf{k} & \textbf{Paired F-Score} \\
\midrule
rule.v & 7 & 0.2000 \\
operate.v & 7 & 0.2222 \\
performance.n & 5 & 0.2833 \\
talk.v & 6 & 0.3117 \\
difference.n & 5 & 0.3146 \\
treat.v & 8 & 0.2492 \\
use.v & 6 & 0.2507 \\
write.v & 9 & 0.2083 \\
degree.n & 7 & 0.2155 \\
play.v & 34 & 0.0495 \\
different.a & 1 & 1.0000 \\
interest.n & 5 & 0.1646 \\
note.v & 3 & 0.3684 \\
hear.v & 5 & 0.2953 \\
judgment.n & 7 & 0.2783 \\
paper.n & 7 & 0.2259 \\
watch.v & 5 & 0.4216 \\
source.n & 9 & 0.1756 \\
express.v & 7 & 0.2684 \\
eat.v & 6 & 0.2938 \\
organization.n & 7 & 0.1779 \\
simple.a & 5 & 0.1481 \\
plan.n & 3 & 0.6459 \\
shelter.n & 5 & 0.2942 \\
suspend.v & 6 & 0.1154 \\
atmosphere.n & 6 & 0.2059 \\
wash.v & 13 & 0.0738 \\
win.v & 4 & 0.4068 \\
miss.v & 8 & 0.2000 \\
provide.v & 7 & 0.3615 \\
bank.n & 9 & 0.1538 \\
party.n & 5 & 0.2740 \\
produce.v & 7 & 0.2641 \\
climb.v & 6 & 0.2458 \\
image.n & 9 & 0.1523 \\
begin.v & 8 & 0.1604 \\
expect.v & 6 & 0.2640 \\
receive.v & 13 & 0.0588 \\
mean.v & 6 & 0.2113 \\
smell.v & 4 & 0.2947 \\
\bottomrule
\end{longtable}
\textbf{Average Paired F-Score:} 0.2070

\noindent
\textbf{Discussion:}  
As expected, random clustering provides the weakest baseline. Since no semantic or contextual information guides the grouping, the results fluctuate randomly and the average F-score remains low (0.207). This serves as a control for evaluating the performance of subsequent methods.

\subsection*{Task 3.2: Cluster with Sparse Representations (6 points)}

\begin{longtable}{@{}lcc@{}}
\caption{Task 3.2: Sparse Representations (Dev Data)}\\
\toprule
\textbf{Target} & \textbf{k} & \textbf{Paired F-Score} \\
\midrule
rule.v & 7 & 0.2253 \\
operate.v & 7 & 0.1994 \\
performance.n & 5 & 0.2580 \\
talk.v & 6 & 0.3037 \\
difference.n & 5 & 0.3158 \\
treat.v & 8 & 0.2438 \\
use.v & 6 & 0.2837 \\
write.v & 9 & 0.1955 \\
degree.n & 7 & 0.1756 \\
play.v & 34 & 0.0590 \\
different.a & 1 & 1.0000 \\
interest.n & 5 & 0.1544 \\
note.v & 3 & 0.6400 \\
hear.v & 5 & 0.2994 \\
judgment.n & 7 & 0.2201 \\
paper.n & 7 & 0.2711 \\
watch.v & 5 & 0.3230 \\
source.n & 9 & 0.1753 \\
express.v & 7 & 0.2605 \\
eat.v & 6 & 0.3494 \\
organization.n & 7 & 0.1817 \\
simple.a & 5 & 0.2400 \\
plan.n & 3 & 0.3621 \\
shelter.n & 5 & 0.3400 \\
suspend.v & 6 & 0.1538 \\
atmosphere.n & 6 & 0.2597 \\
wash.v & 13 & 0.1124 \\
win.v & 4 & 0.4116 \\
miss.v & 8 & 0.2069 \\
provide.v & 7 & 0.3131 \\
bank.n & 9 & 0.1622 \\
party.n & 5 & 0.2514 \\
produce.v & 7 & 0.3122 \\
climb.v & 6 & 0.2609 \\
image.n & 9 & 0.1566 \\
begin.v & 8 & 0.1579 \\
expect.v & 6 & 0.3032 \\
receive.v & 13 & 0.0355 \\
mean.v & 6 & 0.2650 \\
smell.v & 4 & 0.3596 \\
\bottomrule
\end{longtable}
\textbf{Average Paired F-Score:} 0.2140

\noindent
\textbf{Data Analysis:}  
\begin{itemize}
    \item This experiment used a sparse co-occurrence representation where most values are zero.
    \item The model used coocvec-500mostfreq-window-3.filter.magnitude with a symmetric window of size 3. Each word vector encodes co-occurrence frequency within ±3 tokens.
    \item  Three clustering algorithms were compared—K-Means, Agglomerative, and DBSCAN—from scikit-learn. K-Means yielded the most interpretable clusters, while DBSCAN tended to collapse items and Agglomerative over-segmented them.
    \item On the dev set, the F-score improved from 0.207 (random) to 0.214 (sparse), confirming moderate gains from contextual information.
\end{itemize}
  
  


\subsection*{Task 3.3: Cluster with Dense Representations (8 points)}

\begin{longtable}{@{}lcc@{}}
\caption{Task 3.3.1: Dense Representations Results}\\
\toprule
\textbf{Target} & \textbf{k} & \textbf{Paired F-Score} \\
\midrule
rule.v & 7 & 0.2318 \\
operate.v & 7 & 0.2019 \\
performance.n & 5 & 0.2627 \\
talk.v & 6 & 0.3086 \\
difference.n & 5 & 0.2817 \\
treat.v & 8 & 0.2150 \\
use.v & 6 & 0.4468 \\
write.v & 9 & 0.2151 \\
degree.n & 7 & 0.2077 \\
play.v & 34 & 0.0656 \\
different.a & 1 & 1.0000 \\
interest.n & 5 & 0.2536 \\
note.v & 3 & 0.5333 \\
hear.v & 5 & 0.2689 \\
judgment.n & 7 & 0.2063 \\
paper.n & 7 & 0.2408 \\
watch.v & 5 & 0.2636 \\
source.n & 9 & 0.1381 \\
express.v & 7 & 0.2097 \\
eat.v & 6 & 0.2239 \\
organization.n & 7 & 0.2300 \\
simple.a & 5 & 0.2000 \\
plan.n & 3 & 0.3606 \\
shelter.n & 5 & 0.3569 \\
suspend.v & 6 & 0.2154 \\
atmosphere.n & 6 & 0.2744 \\
wash.v & 13 & 0.1796 \\
win.v & 4 & 0.3537 \\
miss.v & 8 & 0.1667 \\
provide.v & 7 & 0.3333 \\
bank.n & 9 & 0.0909 \\
party.n & 5 & 0.2573 \\
produce.v & 7 & 0.2329 \\
climb.v & 6 & 0.2569 \\
image.n & 9 & 0.1855 \\
begin.v & 8 & 0.1989 \\
expect.v & 6 & 0.3357 \\
receive.v & 13 & 0.0773 \\
mean.v & 6 & 0.3028 \\
smell.v & 4 & 0.2750 \\
\bottomrule
\end{longtable}
\textbf{Average Paired F-Score:} 0.2156

\noindent
\textbf{Data Analysis:}
\begin{itemize}
  \item Dense Representation: Filled Matrix 
  \item I used pre-trained dense word embeddings from GoogleNews-vectors-negative300.magnitude, a 300-dimensional Word2Vec model trained on 100 billion tokens from Google News.
  \item Each vector encodes distributed semantic information rather than explicit co-occurrence counts. I used K-means clustering to allow for proper analysis between the gold labeled clusters and the test set clusters.
  \item The dense model achieved an average paired F-score of 0.2156, slightly higher than the sparse baseline .2140. This modest gain reflects that distributed embeddings capture semantic similarity between paraphrases more effectively, though they sometimes overgeneralize frequent verbs.
\end{itemize}  


\subsection*{Task 3.3.3: Error Analysis}

To compare systems, we computed $\Delta = F_{dense} - F_{sparse}$ per target word.

\begin{longtable}{@{}lccc@{}}
\caption{Sample of Sparse vs. Dense F-Score Differences}\\
\toprule
\textbf{Target} & \textbf{$F_{sparse}$} & \textbf{$F_{dense}$} & \textbf{Difference} \\
\midrule
suspend.v & 0.2069 & 0.4906 & +0.2837 \\
bank.n & 0.3373 & 0.5797 & +0.2424 \\
interest.n & 0.2215 & 0.4357 & +0.2142 \\
party.n & 0.3277 & 0.5074 & +0.1797 \\
rule.v & 0.2428 & 0.3900 & +0.1473 \\
\bottomrule
\end{longtable}

\noindent
\textbf{Frequency Analysis:}  
Suspend, bank, interest, and party had the largest positive deltas, indicating that dense embeddings handle abstract and polysemous words better. Sparse models performed better on more syntactically grounded verbs (e.g., talk, treat, use). This pattern suggests dense vectors better capture conceptual relations, while sparse vectors retain context-specific distinctions.

\subsection*{Task 3.4: Cluster Without Predefined $k$ (6 points)}

\begin{longtable}{@{}lcc@{}}
\caption{Task 3.4: Clustering Without Predefined $k$ (Dev Data)}\\
\toprule
\textbf{Target} & \textbf{k} & \textbf{Paired F-Score} \\
\midrule
rule.v & 7 & 0.1891 \\
operate.v & 7 & 0.2809 \\
performance.n & 5 & 0.3140 \\
talk.v & 6 & 0.3634 \\
difference.n & 5 & 0.3521 \\
treat.v & 8 & 0.2012 \\
use.v & 6 & 0.2822 \\
write.v & 9 & 0.2339 \\
degree.n & 7 & 0.3519 \\
play.v & 34 & 0.0647 \\
different.a & 1 & 1.0000 \\
interest.n & 5 & 0.2139 \\
note.v & 3 & 0.4103 \\
hear.v & 5 & 0.2812 \\
judgment.n & 7 & 0.2391 \\
paper.n & 7 & 0.2605 \\
watch.v & 5 & 0.2727 \\
source.n & 9 & 0.1591 \\
express.v & 7 & 0.1991 \\
eat.v & 6 & 0.2592 \\
organization.n & 7 & 0.2416 \\
simple.a & 5 & 0.0952 \\
plan.n & 3 & 0.6142 \\
shelter.n & 5 & 0.3165 \\
suspend.v & 6 & 0.1509 \\
atmosphere.n & 6 & 0.2069 \\
wash.v & 13 & 0.0517 \\
win.v & 4 & 0.2897 \\
miss.v & 8 & 0.1522 \\
provide.v & 7 & 0.3386 \\
bank.n & 9 & 0.1739 \\
party.n & 5 & 0.2319 \\
produce.v & 7 & 0.2775 \\
climb.v & 6 & 0.2451 \\
image.n & 9 & 0.1930 \\
begin.v & 8 & 0.1744 \\
expect.v & 6 & 0.2067 \\
receive.v & 13 & 0.0985 \\
mean.v & 6 & 0.2731 \\
smell.v & 4 & 0.3711 \\
\bottomrule
\end{longtable}
\textbf{Average Paired F-Score:} 0.2129

\noindent
\textbf{Data Analysis:}  
\begin{itemize}
    \item No Cluster K Representation
    \item Like in my previous response, I used pre-trained dense word embeddings from GoogleNews-vectors-negative300.magnitude, a 300-dimensional Word2Vec model trained on 100 billion tokens from Google News. Each vector encodes distributed semantic information rather than explicit co-occurrence counts.
    \item In this experiment, I removed the dependence on the predefined number of clusters (k). For each target word, the algorithm estimated cluster structure automatically using KMeans with a heuristic based on the number of paraphrases, or using DBSCAN for density-based grouping.
    \item The resulting average paired F-score 0.2129 is comparable to the sparse model 0.2140, suggesting that automatic estimation of k does not significantly harm performance. However, some words with many paraphrases (play.v, receive.v) suffered from over-clustering while small sets (plan.n) produced artificially high scores.

\end{itemize}

\section*{Conclusion}
Sparse co-occurrence vectors remain interpretable and competitive, while dense word embeddings offer modest gains in capturing human-like semantic judgments. Increasing embedding dimensionality and corpus size consistently improves Kendall’s $\tau$ correlation, but clustering improvements are less pronounced. Both approaches highlight the tradeoff between context sensitivity and generalization in modern vector space semantics.


\end{document}
